\documentclass[activite]{mathsRC}
\author{Romain Cessac}
\title{Boucle TANT QUE}
\date{Année scolaire 2021/2022}
\numero{}  
\classe{$2^{\text{nde}}$}
\duree{}

%%%%%%%%%%%%%%%%%%%%%%%%%%%%%%%%%%%%%
\begin{document}
\maketitle
%%%%%%%%%%%%%
\section*{Découverte}
\begin{enumerate}
\item
\begin{enumerate}
\item
Recopier sur Capytale \capytale{5434-451217} le programme suivant puis l'exécuter :

\begin{python*}
from random import randint
nb=randint(1,6)
print('Un dé cubique équilibré vient d être lancé')
n=int(input('Devine le résultat : '))
while n!=nb:
	n=int(input('Non, recommence : '))
	
print('Oui, c est bien ça !')
\end{python*}
\item
Quel est le rôle de ce programme ?
\end{enumerate}

\item
\begin{enumerate}
\item
Recopier sur Capytale le programme suivant puis l'exécuter :

\begin{python*}
reponse=str(input('Veux-tu jouer ? Répond par oui ou non.'))
n=0
while reponse=='oui':
    n=n+1
    print(n)
    reponse=str(input('Veux-tu encore jouer ?'))
print('Tu as répondu',n,'fois oui.')
\end{python*}
\item
Quel est le rôle de ce programme ?
\end{enumerate}

\item
Quel semble être le rôle de l'instruction Python suivante ?

\begin{python*}
while <condition>:
	<instruction>
\end{python*}

\item
Compléter l'algorithme suivant de sorte que le programme de la question 1.(a) en soit une implémentation.

\begin{algorithme}
VARIABLES: n,nb (.........................)
DÉBUT
nb = ..................
AFFICHER '.................................................'
AFFICHER '.................................................'
SAISIR ...................
TANT QUE ........................................ FAIRE
	AFFICHER '.................................'
	SAISIR ....................
FINTANTQUE
AFFICHER '.................................................'
FIN
\end{algorithme}
\end{enumerate}

\newpage
\section*{Exercices}

\begin{exercice}[Optimisation d'inégalité]
Déterminer à l'aide d'un programme Python le plus petit entier naturel $n$ tel que :
\[
\np{0.9}^n\leq 10^{-20}.
\]
\end{exercice}

\begin{exercice}[Distance Terre-Lune]
L'épaisseur d'une feuille de papier est de $\np{0.01}$ cm. Cécilia affirme « si l'on pouvait la plier en deux indéfiniment, l'épaisseur de la feuille de papier dépasserait \textit{rapidement} la distance Terre-Lune ».

À l'aide d'un programme Python, que pouvez-vous en dire ?
\end{exercice}

\begin{exercice}[Résolution algorithmique d'équation]
On admet que l'équation $x^3-200x^2+400x=792$ admet dans $\mathbb{R}$ une unique solution. Cette solution est entière et comprise entre $0$ et $\np{1000}$.
\begin{enumerate}
\item
Compléter l'algorithme suivant afin de pouvoir déterminer cette solution.

\begin{algorithme}
VARIABLES: n (.........................)
DÉBUT
n = ..................
TANT QUE ........................................ FAIRE
	n = ..................
FINTANTQUE
AFFICHER '.................................................'
FIN
\end{algorithme}

\item
Implémenter cet algorithme en Python puis donner la solution cherchée.
\end{enumerate}
\end{exercice}


%%%%%%%%%%%%%%%%%%%%%%%%%

\end{document}