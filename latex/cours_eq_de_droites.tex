\documentclass[cours,couleur]{mathsRC}
\author{Romain Cessac}
\title{ÉQUATIONS DE DROITES}
\date{Année scolaire 2022/2023}
\numero{11} 
\classe{2GT 9}
\duree{}
%%%%%%%%%%%%%%%%%%%%%%%%%%%%%%%%%%%%%
\begin{document}
\maketitle
%%%%%%%%%%%%%
\section{Vecteur directeur d'une droite}

\begin{definition}
Soit $d$ une droite passant par deux points distincts $A$ et $B$. On appelle \textbf{vecteur directeur} de la droite $d$ tout vecteur non nul $\vv{u}$ colinéaire au vecteur $\vv{AB}$.
\begin{center}
\begin{tikzpicture}[line cap=round,line join=round,>=triangle 45,scale=0.4]
\draw [line width=1pt,dotted,color=red] (-9.492744370847968,-4.668913026889008)-- (9.815154309739508,5.036115410443252) node[color=red,right]{$d$};
\draw [->,line width=1.3pt] (-4.199965458406945,-2.0085215100535927) -- (5.433703382222296,2.8337988066647735);
\draw (0.4,-1) node{$\vv{AB}$};
\draw [->,line width=1.3pt,color=blue] (2.2267459687850177,5.444535526808458) -- (-0.7257208157053601,3.9604913758741525);
\draw (1.3,4.2) node[color=blue]{$\vv{u}$};
\end{tikzpicture}
\end{center}
\end{definition}

\begin{exemples}
\begin{itemize}[leftmargin=*]
\item
Dans un repère, l'axe des abscisses admet pour vecteurs directeurs des vecteurs à coordonnées $\coorddeux{1}{0}$, $\coorddeux{2}{0}$ ou $\coorddeux{-1}{0}$ \dots
\item
Soit $d$ passant par $A(5;2)$ et $B(-1;3)$. $\vv{AB}$ est un vecteur directeur de $d$ dont on peut déterminer les coordonnées: $\vv{AB}\coorddeux{-1-5}{3-2}$, c'est-à-dire, $\vv{AB}\coorddeux{-6}{1}$.

On peut donc donner d'autres vecteurs directeurs de $d$, colinéaires à $\vv{AB}$, comme $\vv{u}\coorddeux{-12}{2}$ ou $\vv{v}\coorddeux{3}{-\dfrac{1}{2}}$.
\end{itemize}
\end{exemples}

\begin{propriete}
Une droite peut être définie à partir d'un \textbf{vecteur directeur} $\vv{u}$ et d'un \textbf{point} $A$ par lequel elle passe. Ainsi, $M\in d \ssi  \vv{AM}$ et $\vv{u}$ sont colinéaires.
\end{propriete}

\section{Équation cartésienne d'une droite}

\begin{propriete}
Soit $d$ une droite. Dans un repère du plan, il existe $a$, $b$ et $c$ des nombres réels (avec $(a;b)\neq (0;0)$) tels que si $M$ est un point de coordonnées $(x;y)$:
\[
M\in d \ssi ax+by+c=0
\]
Cette équation est appelée \textbf{équation cartésienne de d}.
\end{propriete}

\begin{demonstration}
Nous allons nous ramener à la propriété précédente qui donne aussi une caractérisation d'appartenance à $d$. On sait que $d$ est définie par un vecteur directeur $\vv{u}\coorddeux{\gamma}{\delta}$ et $A(\alpha;\beta)$ un point du plan. Ainsi, si $M(x;y)$, alors $\vv{AM}\coorddeux{x-\alpha}{y-\beta}$. 

On utilise la propriété précédente et le critère de colinéarité par déterminant :
\begin{align*}
M\in d &\ssi  \vv{AM}\text{ et }\vv{u}\text{ sont colinéaires}\\
&\ssi \det (\vv{AM}, \vv{u})=0\\
&\ssi (x-\alpha)\delta-(y-\beta)\gamma=0\\
&\ssi \delta x - \gamma y +(\beta \gamma - \alpha \delta)=0\\
\end{align*}
Notre propriété est bien démontrée en prenant $a= \delta$, $b=- \gamma$ et $c= \beta \gamma - \alpha \delta$. 

Notons bien que $a$ et $b$ ne sont jamais nuls simultanément car $\vv{u}\neq\vv{0}$.
\end{demonstration}

\begin{remarque}
Il existe une \textbf{infinité d'équations cartésiennes} pour une même droite. Elles sont toutes équivalentes en appliquant un même coefficient de proportionnalité (non nul) aux trois paramètres $a$, $b$ et $c$.
\end{remarque}

\begin{exemple}
Soit $d$ la droite passant par $A(3;2)$ et $B(0;-3)$. Déterminons une équation cartésienne de $d$.

$\vv{AB}\coorddeux{-3}{-5}$ est un vecteur directeur de $d$. Soit $M(x;y)$ un point du plan. Ainsi, $\vv{AM}\coorddeux{x-3}{y-2}$.
\begin{align*}
M\in d &\ssi \det(\vv{AM},\vv{AB})=0\\
&\ssi -5(x-3)-(-3)(y-2)=0\\
&\ssi -5x+15+3y-6=0\\
&\ssi -5x+3y+9=0
\end{align*}
\end{exemple}

\begin{theoreme}[Droites parallèles]
Soient $d$ et $d'$ deux droites d'équations cartésiennes respectives $ax+by+c=0$ et $a'x+b'y+c'=0$.
\[
d\ //\ d' \ssi ab'-ba'=0
\]
\end{theoreme}

\begin{demonstration} Supposons que $b\neq 0$ et $b'\neq 0$ (si l'un des deux est nul, c'est trivial).

\bigskip
Soient $M_{1}\left(-1;\dfrac{a-c}{b}\right)$, $M_{2}\left(1;\dfrac{-a-c}{b}\right)$, $M'_{1}\left(-1;\dfrac{a'-c'}{b'}\right)$ et $M'_{2}\left(1;\dfrac{-a'-c'}{b'}\right)$.

\bigskip
On peut affirmer que $M_{1}$ et $M_{2}$ appartiennent à $d$ mais aussi que $M'_{1}$ et $M'_{2}$ appartiennent à $d'$. Pour $M_{1}$ par exemple, on vérifie que ses coordonnées sont compatibles avec l'équation cartésienne $ax+by+c=0$. C'est le cas : $a\times(-1)+b\dfrac{a-c}{b}+c=-a+a-c+c=0$.

On donne ainsi des vecteurs directeurs $\vv{u}$ et $\vv{v}$ de $d$ et $d'$.
\[
\vv{u}=\vv{M_{1}M_{2}}=\coorddeux{2}{-\dfrac{2a}{b}}\text{ et }\vv{v}=\vv{M'_{1}M'_{2}}=\coorddeux{2}{-\dfrac{2a'}{b'}}
\]
Ainsi, $d\ //\ d' \ssi 2\times \left( -\dfrac{2a'}{b'}\right)- 2\times \left(-\dfrac{2a}{b}\right)=0 \ssi -4\dfrac{a'}{b'}+4\dfrac{a}{b}=0 \ssi -a'b+ab'=0$ (nous avons multiplié par $\dfrac{bb'}{4}\neq 0$).
\end{demonstration}

\begin{exemple}
Soient $d$ et $d'$ d'équations cartésiennes respectives $21x-3y+24=0$ et $-7x+y+2=0$.

$d$ et $d'$ sont parallèles car $21\times 1 - (-3)\times (-7)=0$.
\end{exemple}

\section{Équation réduite d'une droite}

\begin{propriete}
Soit $d$ une droite d'équation cartésienne $ax+by+c=0$ ($(a;b)\neq (0;0)$).
\begin{itemize}[leftmargin=*]
\item
Si $b=0$, alors $ax+by+c=0$ est équivalente à une \textbf{unique} équation de la forme $x=k$ appelée \textbf{équation réduite de $d$}, où $k\in \R$.
\begin{center}
\begin{tikzpicture}[>=triangle 45,scale=0.8]
\begin{axis}[
axis lines=middle,
grid style=dashed,
xmin=-12.39494337712897,
xmax=24.816377941378718,
ymin=-12.708158288294102,
ymax=13.11338597264735,
xtick={-100,100},
ytick={-100,100},]
\clip(-12.39494337712897,-12.708158288294102) rectangle (24.816377941378718,13.11338597264735);
\draw [line width=2pt,color=blue] (5,-12.708158288294102) -- (5,13.11338597264735);
\draw (10,5) node[color=blue,scale=1.3]{$x=k$};
\end{axis}
\end{tikzpicture}
\end{center}
\item
Si $b\neq 0$, alors $ax+by+c=0$ est équivalente à une \textbf{unique} équation de la forme $y=mx+p$ appelée \textbf{équation réduite de $d$}, où $m\in \R$ est le \textbf{coefficient directeur de $d$} et $p\in\R$ l'\textbf{ordonnée à l'origine de $d$}.
\begin{center}
\begin{tikzpicture}[>=triangle 45,scale=0.8]
\begin{axis}[
axis lines=middle,
grid style=dashed,
xmin=-12.39494337712897,
xmax=24.816377941378718,
ymin=-12.708158288294102,
ymax=13.11338597264735,
xtick={-100,100},
ytick={-100,100},]
\clip(-12.39494337712897,-12.708158288294102) rectangle (24.816377941378718,13.11338597264735);
\draw [line width=2pt,domain=-12.39494337712897:24.816377941378718,color=red] plot(\x,{(--2-0.5*\x)/1});
\draw (10,5) node[color=red,scale=1.3]{$y=mx+p$};
\end{axis}
\end{tikzpicture}
\end{center}
\end{itemize}
\end{propriete}

\begin{demonstration}
\begin{itemize}[leftmargin=*]
\item
Si $b=0$, alors $a\neq 0$ et pour tout point $M(x;y)$ de $d$, on a :
\[
ax+by+c=0 \ssi ax+c=0 \ssi x=-\dfrac{c}{a}
\]
C'est-à-dire, $k=\dfrac{c}{a}$.
\item
Si $b\neq 0$, pour $M(x;y)$ de $d$:
\[
ax+by+c=0 \ssi y=\dfrac{-ax-c}{b} \ssi y=\dfrac{-a}{b}x+\dfrac{-c}{b} 
\]
C'est-à-dire, $m=-\dfrac{a}{b}$ et $p=-\dfrac{c}{b}$.\qedhere
\end{itemize}
\end{demonstration}

\begin{exemples}
\begin{itemize}[leftmargin=*]
\item
Soit $d$ d'équation cartésienne $6x+20=0$.

Nous sommes dans le premier cas, on isole $x$ et donc l'équation réduite de $d$ est $x=-\dfrac{20}{6}$ ou plutôt $x=-\dfrac{10}{3}$.
\item
Soit $d$ d'équation cartésienne $\dfrac{2}{3}x-\dfrac{5}{7}y=0$. C'est le second cas, on isole $y$ et ainsi:
\begin{align*}
\dfrac{2}{3}x-\dfrac{5}{7}y=0 &\ssi \dfrac{5}{7}y=\dfrac{2}{3}x\\
&\ssi y=\dfrac{7}{5}\times \dfrac{2}{3}x\\
&\ssi y=\dfrac{14}{15}x.
\end{align*}
\end{itemize}
\end{exemples}

\begin{remarque}
Si l'équation réduite d'une droite est sous la deuxième forme $y=mx+p$, cette droite est la \textbf{représentation graphique d'une fonction affine}: c'est ainsi cohérent d'utiliser le même vocabulaire. Le coefficient directeur et l'ordonnée à l'origine peuvent donc aussi être déterminés graphiquement mais il est aussi souvent plus simple de tracer la droite qu'à partir de l'équation cartésienne.
\end{remarque}

\begin{theoreme}[Droites parallèles]
Soient $d$ et $d'$ d'équations réduites respectives $y=mx+p$ et $y=m'x+p'$.
\[
d\ //\ d' \ssi m=m'
\]
\end{theoreme}

\begin{demonstration}
On se ramène au résultat sur les équations cartésiennes. $d$ et $d'$ ont pour équations cartésiennes $ax+by+c=0$ et $a'x+b'y+c'=0$ avec $b\neq 0$ et $b'\neq 0$. 

\bigskip
On sait que $d\ //\ d' \ssi ab'-ba'=0$ et nous avons déjà vu que $m=-\dfrac{a}{b}$ et $m'=-\dfrac{a'}{b'}$. Donc, comme $bb'\neq 0$:
\[
d\ //\ d' \ssi ab'-ba'=0 \ssi \dfrac{a}{b}-\dfrac{a'}{b'}=0 \ssi -m+m'=0 \ssi m=m'.
\]
\end{demonstration}

\begin{exemple}
Soient $f$, $g$ et $h$ trois fonctions affines définies sur $\R$ par : $f(x)=3x+1$, $g(x)=2x+1$ et $h(x)=3x-10$.
Les représentations graphiques de $f$ et $h$ sont parallèles (même coefficient directeur) mais pas celles de $f$ et $g$ ou celles de $g$ et $h$.
\begin{center}
\begin{tikzpicture}[>=triangle 45,scale=0.8]
\begin{axis}[
axis lines=middle,
grid style=dashed,
xmin=-8.39494337712897,
xmax=10.816377941378718,
ymin=-12.708158288294102,
ymax=13.11338597264735,
xtick={-100,100},
ytick={-100,100},]
\clip(-12.39494337712897,-12.708158288294102) rectangle (24.816377941378718,13.11338597264735);
\draw [line width=2pt,domain=-12.39494337712897:24.816377941378718,color=red] plot(\x,{3*\x+1)});
\draw (-3.5,5) node[color=red,scale=1.3]{$y=3x+1$};
\draw [line width=2pt,domain=-12.39494337712897:24.816377941378718,color=blue] plot(\x,{3*\x-10)});
\draw (7,-4) node[color=blue,scale=1.3]{$y=3x-10$};
\draw [line width=2pt,domain=-12.39494337712897:24.816377941378718,color=magenta] plot(\x,{2*\x+1)});
\draw (-2,9) node[color=magenta,scale=1.3]{$y=2x+5$};

\end{axis}
\end{tikzpicture}
\end{center}
\end{exemple}

\section{Systèmes d'équations}

\begin{definition}[Systèmes de deux équations à deux inconnues]
On appelle \textbf{système de deux équations à deux inconnues} $x$ et $y$ tout système pouvant s'écrire sous la forme suivante (où $a$, $b$, $c$, $a'$, $b'$, $c'$ $\in\R$).
\[
\systeme{ax+by=c,a'x+b'y=c'}
\]
Une \textbf{solution} de ce système est un couple $(x;y)$ solution des deux équations simultanément.
\end{definition}

\begin{exemple}
On considère le système \systeme{x+y=0,3x-y=4}.

\bigskip
Le couple $(1;-1)$ est solution de $x+y=0$ ainsi que de $3x-y=3$ donc $(1;-1)$ est solution du système.
\end{exemple}

\begin{remarque}
Si $(a;b)\neq(0;0)$ et $(a';b')\neq(0;0)$, nous obtenons des \textbf{équations cartésiennes} de droites $d$ et $d'$.
\[
\systeme{ax+by=c,a'x+b'y=c'} \ssi \systeme{ax+by-c=0,a'x+b'y-c'=0}
\]
Ainsi, une solution du système correspond à un point $M(x;y)$ appartenant à $d$ mais aussi à $d'$. C'est-à-dire, $M$ appartient à l'\textbf{intersection des deux droites} $d$ et $d'$.
\end{remarque}

\begin{propriete}[Existence et unicité de la solution]
Soit un système \systeme{ax+by=c,a'x+b'y=c'} tel que $(a;b)\neq(0;0)$ et $(a';b')\neq(0;0)$.

Il existe un \textbf{unique couple solution} si et seulement si $ab'-ba'\neq0$.
\end{propriete}

\begin{demonstration}
Les deux équations du système correspondent à deux équations cartésiennes de droites qu'on note $d$ et $d'$. On peut décomposer la situation en deux cas disjoints:
\begin{itemize}
\item
$d$ et $d'$ sont sécantes en un seul point.
\item
$d$ et $d'$ sont parallèles (peut-être même confondues).
\end{itemize}
Or nous avons vu que $d\ //\ d' \ssi ab'-ba'=0$. C'est-à-dire, en passant à la négation, $d$ et $d'$ sont sécantes en un seul point $\ssi ab'-ba'\neq 0$
\end{demonstration}

\begin{exemples}
\begin{itemize}[leftmargin=*]
\item
\systeme*{16x+12y+40=0,2x+2y=0} \textbf{admet} une unique solution car $16\times 2 -2\times12=32-24=8\neq 0$.
\item
\systeme{25x+12y-1=0,50x+24y-2=0} \textbf{n'admet pas} une unique solution car $25\times 24 -50\times12=0$.

Il y a, en fait, une infinité de solutions car les équations cartésiennes désignent la même droite.
\end{itemize}
\end{exemples}

\begin{methode}[Résolution d'un système à deux équations et deux inconnues]
Donnons les différentes techniques permettant de manipuler des systèmes équivalents (c'est-à-dire dont les solutions sont identiques). 

Résoudre un système \systeme{ax+by=c,a'x+b'y=c'} , c'est trouver $\alpha,\beta\in\R$ tels que :
\[
\systeme{ax+by=c,a'x+b'y=c'} \ssi \systeme*{x=\alpha,y=\beta}.
\]
\begin{description}
\item[Opérations usuelles:]
Il est possible de remplacer une des équations par une équation qui lui est équivalente (à l'aide des techniques déjà connues comme factoriser, développer, ajouter un même nombre, multiplier par un nombre non nul). 

Par exemple, 
\[
\systeme*{16x+12y+40=0,2x+2y=0} \ssi \systeme*{4x+3y+10=0,x+y=0} \ssi \systeme*{4x+3y=-10,x+y=0}.
\]
\item[Permutation:] On peut échanger indifféremment les deux équations du système : 
\[
\systeme{ax+by=c,a'x+b'y=c'} \ssi \systeme{a'x+b'y=c',ax+by=c}.
\]
\item[Substitution:] Si on a isolé une des inconnues, on peut la substituer dans la seconde équation.

Par exemple,
\[
\systeme*{4x+3y=-10,x+y=0} \ssi \systeme*{4x+3y=-10,x=-y} \ssi \systeme*{4(\-y)\+3y= -10,x= -y} \ssi \systeme*{-y=-10,x=-y}
\]
Il est très facile de conclure, en utilisant à nouveau la technique de substitution en remplaçant $y$ par sa valeur $10$.
\[
\systeme*{-y=-10,x=-y} \ssi \systeme*{y=10,x=-y} \ssi \systeme*{y=10,x=-10}
\]
Finalement, $\systeme*{16x+12y+40=0,2x+2y=0} \ssi \systeme*{y=10,x=-10} \ssi \systeme*{x=-10,y=10}$.

L'unique solution du système \systeme*{16x+12y+40=0,2x+2y=0} est le couple $(-10;10)$.
\item[Combinaison linéaire:]
On peut ajouter ou soustraire (membre à membre) un multiple non nul d'une équation à l'autre.

Ainsi, on soustrait $6$ fois la deuxième équation à la première.
\begin{align*}
\systeme*{16x+12y+40=0,2x+2y=0} &\ssi \systeme*{16x\+12y\+40\-6\times(2x\+2y)=0-6\times 0,2x+2y=0} \\
&\ssi \systeme*{4x+0y+40=0,2x+2y=0}\\
&\ssi \systeme*{4x+40=0,2x+2y=0}\\
&\ssi \systeme*{x+10=0,x+y=0} \ \text{On divise les équations par $4$ et $2$ respectivement.}\\
&\ssi \systeme*{x=-10,x\+y\-x=-(-10)} \ \text{On soustrait la première équation à la deuxième.}\\
&\ssi \systeme*{x=-10,y=10}\\
\end{align*}
\end{description}
\end{methode}

\begin{remarque}
Les techniques de \textbf{substitution} ou \textbf{combinaison linéaire} sont fondamentales et doivent être choisies judicieusement selon la forme du système à résoudre.
\end{remarque}

%%%%%%%%%%%%%
\end{document}